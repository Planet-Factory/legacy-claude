\section{History of the Algorithms} \label{sec:history}
Back when I was a young naive programmer, I made a thing. Now a few years down the line I made the thing again, but infinitely better. So I have no use for the old thing anymore. But fear not,
old algorithms (used by CLAuDE) will be collected here. This is just for historical purposes.

\subsection{Radiation}
\subsubsection{Adding Layers}
Remember \autoref{eq:atmos change}? We need this equation for every layer in the atmosphere. This also means that we have to adjust the main calculation of the code, which is described  in 
\autoref{alg:temperature with density}. The $T_a$ needs to change, we need to either add a dimension (to indicate which layer of the atmosphere we are talking about) or we need to add different
matrices for each atmosphere layer. We opt for adding a dimension as that costs less memory than defining new arrays 
\footnote{This has to do with pointers, creating a new object always costs a bit more space than adding a dimension as we need a pointer to the object and what type of object it is whereas with 
adding a dimension we do not need this additional information as it has already been defined}. So $T_a$, and all other matrices that have to do with the atmosphere (so not $T_p$ for instance) 
are no longer indexed by $lat, lon$ but are indexed by $lat, lon, layer$. We need to account for one more thing, the absorbtion of energy from another layer. The new equation is shown in 
\autoref{eq:atmos change layer}. Here $k$ is the layer of the atmosphere, $k = -1$ means that you use $T_p$ and $k = nlevels$ means that $T_{a_{nlevels}} = 0$ as that is space. Also, let us
rewrite the equation a bit such that the variables that are repeated are only written once and stuff that is divided out is removed, which is done in \autoref{eq:atmos change layer improved}.
Let us also clean up the equation for the change in the surface temperature (\autoref{eq:surface change}) in \autoref{eq:surface change improved}.

\begin{subequations}
    \begin{equation}
        \Delta T_{a_k} = \frac{\delta t (\sigma \epsilon_{k - 1}T_{a_{k - 1}}^4 + \sigma \epsilon_{k + 1}T_{a_{k + 1}}^4 - 2\epsilon_k\sigma T_{a_k}^4)}{C_a}
        \label{eq:atmos change layer}
    \end{equation}
    \begin{equation}
        \Delta T_{a_k} = \frac{\delta t \sigma (\epsilon_{k - 1}T_{a_{k - 1}}^4 + \epsilon_{k + 1}T_{a_{k + 1}}^4 - 2\epsilon_kT_{a_k}^4)}{C_a}
        \label{eq:atmos change layer improved}
    \end{equation}
    \begin{equation}
        \Delta T_p = \frac{\delta t (S + \sigma(4\epsilon_pT_a^4 - 4T_p^4))}{4C_p}
        \label{eq:surface change improved}
    \end{equation}
\end{subequations}

With the changes made to the equation, we need to make those changes in the code as well. We need to add the new dimension to all matrices except $T_p$ and $a$ as they are unaffected (with 
regards to the storage of the values) by the addition of multiple atmospheric layers. Every other matrix is affected. The new code can be found in \autoref{alg:temperature layer}. $\delta z$

\begin{algorithm}[hbt]
    \SetAlgoLined
    \SetKwInput{Input}{Input}
    \SetKwInOut{Output}{Output}
    \Input{amount of energy that hits the planet $S$}
    \Output{Temperature of the planet $T_p$, temperature of the atmosphere $T_a$}
    \For{$lat \leftarrow -nlat$ \KwTo $nlat$}{
        \For{$lon \leftarrow 0$ \KwTo $nlot$}{
            \For{$layer \leftarrow 0$ \KwTo $nlevels$}{
                $T_p[lat, lon] \leftarrow T_p[lat, lon] + \frac{\delta t ((1 - a[lat, lon])S + \sigma(4\epsilon[0](T_a[lat, lon, 0])^4 - 4(T_p[lat, lon])^4))}
                {4C_p[lat, lon]}$ \;
                \uIf{$layer = 0$}{
                    $T_a[lat, lon, layer] \leftarrow T_a[lat, lon, layer] + \frac{\delta t \sigma((T_p[lat, lon])^4 - 2\epsilon[layer](T_a[lat, lon, layer])^4)}
                    {\rho[lat, lon, layer]C_a\delta z[layer]}$ \;
                }\uElseIf{$layer = nlevels - 1$}{
                    $T_a[lat, lon, layer] \leftarrow T_a[lat, lon, layer] + \frac{\delta t \sigma(\epsilon[layer - 1](T_a[lat, lon, layer - 1])^4 - 2\epsilon[layer](T_a[lat, lon, layer])^4)}
                    {\rho[lat, lon, layer]C_a\delta z[layer]}$ \;
                }\uElse{
                    $T_a[lat, lon, layer] \leftarrow T_a[lat, lon, layer] + \frac{\delta t \sigma(\epsilon[layer - 1](T_a[lat, lon, layer - 1])^4 + \epsilon[layer + 1]T_a[lat, lon, layer + 1] 
                    - 2\epsilon[layer](T_a[lat, lon, layer])^4)}{\rho[lat, lon, layer]C_a\delta z[layer]}$ \;
                }
            }
        }
    }
    \caption{The main function for the temperature calculations}
    \label{alg:temperature layer}
\end{algorithm}

We also need to initialise the $\epsilon$ value for each layer. We do that in \autoref{alg:epsilon}.

\begin{algorithm}
    $\epsilon[0] \leftarrow 0.75$ \;
    \For{$i \leftarrow 1$ \KwTo $nlevels$}{
        $\epsilon[i] \leftarrow 0.5\epsilon[i - 1]$
    }
    \caption{Intialisation of the insulation of each layer (also known as $\epsilon$)}
    \label{alg:epsilon}
\end{algorithm}

\subsection{Velocity}
\subsubsection{The Primitive Equations and Geostrophy} \label{sec:primitive}
The primitive equations (also known as the momentum equations) is what makes the air move. It is actually kind of an injoke between physicists as they are called the primitive equations but 
actually look quite complicated (and it says $fu$ at the end! \cite{simon}). The primitive equations are a set of equations dictating the direction in the $u$ and $v$ directions as shown in 
\autoref{eq:primitive u} and \autoref{eq:primitive v}. We can make the equations simpler by using and approximation called geostrophy which means that we have no vertical motion, such that the
terms with $\omega$ in \autoref{eq:primitive u} and \autoref{eq:primitive v} become 0. We also assume that we are in a steady state, i.e. there is no acceleration which in turn means that the 
whole middle part of the equations are $0$. Hence we are left with \autoref{eq:primitive u final} and \autoref{eq:primitive v final}.

\begin{subequations}
    \begin{equation}
        \frac{du}{dt} = \frac{\delta u}{\delta t} + u\frac{\delta u}{ \delta x} + v\frac{\delta u}{\delta v} + \omega\frac{\delta u}{\delta p} = -\frac{\delta \Phi}{\delta x} + fv
        \label{eq:primitive u}
    \end{equation}
    \begin{equation}
        \frac{dv}{dt} = \frac{\delta v}{\delta t} + u\frac{\delta v}{ \delta x} + v\frac{\delta v}{\delta v} + \omega\frac{\delta v}{\delta p} = -\frac{\delta \Phi}{\delta y} - fu
        \label{eq:primitive v}
    \end{equation}

    \begin{equation}
        0 = -\frac{\delta \Phi}{\delta x} + fv
        \label{eq:primitive u final}
    \end{equation}
    \begin{equation}
        0 = -\frac{\delta \Phi}{\delta y} - fu
        \label{eq:primitive v final}
    \end{equation}
\end{subequations}

\autoref{eq:primitive u final} can be split up into to parts, the $\frac{\delta \Phi}{\delta x}$ part (the gradient force) and the $fv$ part (the coriolis force). The same applies to 
\autoref{eq:primitive v final}. Effectively we have a balance between the gradient and the coriolis force as shown in \autoref{eq:pu simple} and \autoref{eq:pv simple}. The symbols in both of 
these equations are:

\begin{itemize}
    \item $\Phi$: The geopotential, potential (more explanation in \autoref{sec:potential}) of the planet's gravity field ($Jkg^{-1}$).
    \item $x$: The change in the East direction along the planet surface ($m$).
    \item $y$: The change in the North direction along the planet surface ($m$).
    \item $f$: The coriolis parameter as described by \autoref{eq:coriolis}, where $\Omega$ is the rotation rate of the planet (for Earth $7.2921 \cdot 10^{-5}$) ($rad \ s^{-1}$) and $\theta$ is the 
    latitude \cite{coriolis}.
    \item $u$: The velocity in the latitude ($ms^{-1}$).
    \item $v$: The velocity in the longitude ($ms^{-1}$).
\end{itemize}

\begin{subequations}
    \begin{equation}
        f = 2\Omega\sin(\theta)
        \label{eq:coriolis}
    \end{equation}
    \begin{equation}
        \frac{\delta \Phi}{\delta x} = fv
        \label{eq:pu simple}
    \end{equation}
    \begin{equation}
        \frac{\delta \Phi}{\delta y} = -fu
        \label{eq:pv simple}
    \end{equation}
    \begin{equation}
        \frac{\delta p}{\rho \delta x} = fv
        \label{eq:pu simple final}
    \end{equation}
    \begin{equation}
        \frac{\delta p}{\rho \delta y} = -fu
        \label{eq:pv simple final}
    \end{equation}
\end{subequations}

Since we want to know how the atmosphere moves, we want to get the v and u components of the velocity vector (since $v$ and $u$ are the veolicites in longitude and latitude, if we combine them 
in a vector we get the direction of the overall velocity). So it is time to start coding and calculating! If we look back at \autoref{alg:stream1v2}, we can see that we already have a double 
for loop. In computer science, having multiple loops is generally considered a bad coding practice as you usually can just reuse the indices of the already existing loop, so you do not need to 
create a new one. However this is a special case, since we are calculating new temperatures in the double for loop. If we then also would start to calculate the velocities then we would use new 
information and old information at the same time. Since at index $i - 1$ the new temperature has already been calculated, but at the index $i + 1$ the old one is still there. So in order to fix 
that we need a second double for loop to ensure that we always use the new temperatures. We display this specific loop in \autoref{alg:stream2}. Do note that everything in \autoref{alg:stream1v2} 
is still defined and can still be used, but since we want to focus on the new code, we leave out the old code to keep it concise and to prevent clutter. 

\begin{algorithm}[hbt]
    \SetAlgoLined
    \For{$lat \in [-nlat, nlat]$}{
        \For{$lon \in [0, nlon]$}{
            $u[lat, lon] \leftarrow -\frac{p[lat + 1, lon] - p[lat - 1, lon]}{\delta y} \cdot \frac{1}{f[lat]\rho}$ \;
            $v[lat, lon] \leftarrow \frac{p[lat, lon + 1] - p[lat, lon - 1]}{\delta x[lat]} \cdot \frac{1}{f[lat]\rho}$ \;
        }
    }
    \caption{The main loop of the velocity of the atmosphere calculations}
    \label{alg:stream2}
\end{algorithm}

The gradient calculation is done in \autoref{alg:gradient}. For this to work, we need the circumference of the planet. Herefore we need to assume that the planet is a sphere. While that is not 
technically true, it makes little difference in practice and is good enough for our model. The equation for the circumference can be found in \autoref{eq:circumference} \cite{circumference}, 
where $r$ is the radius of the planet. Here we also use the f-plane approximation, where the coriolis paramter has one value for the northern hemisphere and one value for the southern hemisphere 
\cite{fplane}.

\begin{equation}
    2 \pi r
    \label{eq:circumference}
\end{equation}

\begin{algorithm}
    \SetAlgoLined
    $C \leftarrow 2\pi R$ \;
    $\delta y \leftarrow \frac{C}{nlat}$ \;

    \For{$lat \in [-nlat, nlat]$}{
        $\delta x[lat] \leftarrow \delta y \cos(lat \cdot \frac{\pi}{180})$ \;

        \eIf{$lat < 0$}{
            $f[lat] \leftarrow -10^{-4}$ \;
        }{
            $f[lat] \leftarrow 10^{-4}$ \;
        }
    }
    \caption{Calculating the gradient $\delta x$ (note that this algorithm is obsolete)}
    \label{alg:gradient}
\end{algorithm}

Because of the geometry of the planet and the construction of the longitude latitude grid, we run into some problems when calculating the gradient. Since the planet is not flat ("controversial 
I know"\cite{simon}) whenever we reach the end of the longitude we need to loop around to get to the right spot to calculate the gradients (as the planet does not stop at the end of the 
longitude line but loops around). So to fix that we use the modulus (mod) function which does the looping for us if we exceed the grid's boundaries. We do haveanother problem though, the poles. 
As the latitude grows closer to the poles, they are converging on the center point of the pole. Looping around there is much more difficult so to fix it, we just do not consider that center 
point in the main loop. The changed algorithm can be found in \autoref{alg:stream2v2}

\begin{algorithm}[hbt]
    \SetAlgoLined
    \For{$lat \in [-nlat + 1, nlat - 1]$}{
        \For{$lon \in [0, nlon]$}{
            $u[lat, lon] \leftarrow -\frac{p[(lat + 1) \text{ mod } nlat, lon] - p[(lat -1) \text{ mod } nlat, lon]}{\delta y} \cdot \frac{1}{f[lat]\rho}$ \;
            $v[lat, lon] \leftarrow \frac{p[lat, (lon + 1) \text{ mod } nlon] - p[lat, (lon -1) \text{ mod } nlon]}{\delta x[lat]} \cdot \frac{1}{f[lat]\rho}$ \;
        }
    }
    \caption{The main loop of the velocity of the atmosphere calculations}
    \label{alg:stream2v2}
\end{algorithm}

Do note that the pressure calculation is done between the temperature calculation in \autoref{alg:stream1v2} and the $u, v$ calculations in \autoref{alg:stream2v2}. At this point our model shows
a symmetric vortex around the sun that moves with the sun. This is not very realistic as you usually have convection and air flowing from warm to cold, but we do not have that complexity yet 
(due to our single layer atmosphere).